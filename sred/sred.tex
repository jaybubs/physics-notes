\documentclass{article}
\usepackage[margin=0.5in]{geometry}
\usepackage{amsmath}
\usepackage{physics}
\usepackage{array}
\newcolumntype{C}{>{$}c<{$}} % math-mode version of "c" column type
\begin{document}

\section*{SHM path integral}%
\label{sec:shm_path_integral}

Start with a hammy:

\begin{align}
	H \left(P,Q\right) = \frac{P^2}{2m} + \frac{m\omega^2Q^2}{2}
\end{align}
In path integrals, operators are functions so $P \rightarrow p$ and $Q \rightarrow q$. Here we're interested in gound state to ground state, because of reasons. Using:
\begin{align}
	\braket{0}{0} &= \braket{0}{q_n}\braket{q_{n-1}}{q_{n-2}}\ldots \braket{q_1}{0} \\
		      &\Downarrow \nonumber \\
	\braket{0}{0} &= \int \mathcal{D}p \mathcal{D}q \,exp \left[i \int_{-\infty}^{\infty}dt \left(p\dot{q} - \left(1-i\epsilon)H + fq \right)\right)\right] \label{a1}
\end{align}
Where H is Weyl-ordered (average of normal and anti-normal ordering).\\
Applying $(1-i\epsilon)$ on H will pick out the ground states in $\pm\infty$ time, leads to the following transforms:
\begin{align}
	\frac{1}{2}m\omega^2q &\rightarrow \frac{1}{2}(1-i\epsilon)m\omega^2 \\
		  & and \\
	\frac{1}{2m}p^2 &\rightarrow \frac{1}{2(1-i\epsilon)m}p^2 = \frac{(1-i\epsilon)(1+i\epsilon)}{2m(1+i\epsilon)}p^2 = \frac{1+i\epsilon-i\epsilon+\mathcal{O}(2)}{2m(1+i\epsilon)}p^2 \\
\Rightarrow \frac{1}{2m}p^2 &\rightarrow \frac{1}{2(1+i\epsilon)m}p^2
\end{align}
subbing back into \ref{a1}:
\begin{align}
	\braket{0}{0} &= \int \mathcal{D}p \mathcal{D}q \,exp \left[i \int_{-\infty}^{\infty}dt \left(p\dot{q} - \frac{p^2}{(1+i\epsilon)2m} - \frac{1-i\epsilon}{2} m\omega^2q^2 + fq \right)\right]
\end{align}

now the sweet insides can be integrated out over $\mathcal{D}p$ to turn it into a laggy, using $\partial_{p}\mathcal{H} = \dot{q}$ :
\begin{align}
	\braket{0}{0}=\int \mathcal{D}q \,exp \left[ i \int_{-\infty}^{\infty}dt \left( \frac{1}{2}(1+i\epsilon)m\dot{q}^2 - \frac{1}{2}(1-i\epsilon)m\omega^2q^2 + fq \right) \right] \label{a2}
\end{align}
Next up, perform a fourier transform to get this shit into functions of energies and shit. Use these variables, they're good, trust me, you are me after all:
\begin{align}
	q(t)&=\int_{-\infty}^{\infty} \frac{dE}{\tau} e^{-iEt} \tilde{q}(E) \\
	\dot{q}(t)&=\int_{-\infty}^{\infty} -\frac{dE}{\tau}\,iE\,e^{-iEt} \tilde{q}(E) \\
\tilde{q}(E)&=\int_{-\infty}^{\infty}dt\,e^{iEt}q(t)
\end{align}
Now take all that rubbish and shove it into the terms in \ref{a2} \\ do not forget that there's squared variables so we'll have to integrate over two different variables, thus E and E', and t and t'
\begin{align}
	\braket{0}{0}_f = \int \mathcal{D}q\, exp \left\{ \frac{i}{2} \int_{-\infty}^{\infty} \frac{dE}{\tau} \frac{dE'}{\tau}\, e^{-i(E+E')t} \bigg[ \Big( -(1+i\epsilon)EE' - (1-i\epsilon)\omega^2 \Big) \tilde{q}(E) \tilde{q}(E') + \tilde{f}(E) \tilde{q}(E') + \tilde{f}(E') \tilde{q}(E) \bigg] \right\}
\end{align}
Now it looks like a fucking goddamn mess, but we can integrate over E' using a neat delta function:
\begin{align}
	\tau \delta(a-b)= \int dx\,e^{i(a-b)x} \implies \frac{1}{2} \int \frac{dE\,dE'}{\tau^2} \,\delta(E+E') [ \ldots ]
\end{align}
\newpage Reslutting in:
\begin{align}
	\braket{0}{0}_f = \int \mathcal{D}q\, exp \Bigg\{ \frac{1}{2} \int_{-\infty}^{\infty} \frac{dE}{\tau} \bigg[ &\underbrace{ \Big( (1+i\epsilon)E^2 - (1-i\epsilon)\omega^2 \Big) }_{\downarrow} \tilde{q}(E) \tilde{q}(-E) + \tilde{f}(E) \tilde{q}(-E) + \tilde{f}(-E) \tilde{q}(E) \bigg]  \Bigg\} \label{a4} \\
														     &E^2 - \omega^2 + i(E^2 + \omega^2)\epsilon \implies E^2 - \omega^2 - i\epsilon \label{a3}
\end{align}
Now as a magic trick we do a little quasigauge shift. We gunna introduce x as a shift of q and the inverse of \ref{a3}. The benefit of this is that since it's a linear shift in q/x only the measure won't change:
\begin{align}
	\tilde{x}(E) &= \tilde{q}(E) + \frac{ \tilde{f}(E) }{E^2 - \omega^2 + i\epsilon } \\
	\mathcal{D}q &= \mathcal{D}x
\end{align}
Substituting back into \ref{a4} and splitting the path integral into a phase dependent on x and independent of x:
\begin{align}
	\braket{0}{0}_f = exp \bigg[ \frac{i}{2} \int \frac{dE}{\tau} \frac{ \tilde{f}(E) \tilde{f}(-E) }{-E^2+\omega^2-i\epsilon} \bigg] \cdot \underbrace{ \int \mathcal{D}x\, exp \bigg[ \frac{i}{2} \int \frac{dE}{\tau} \tilde{x}(E) (E^2 - \omega^2 +i\epsilon) \tilde{x}(-E) \bigg] }_{\text{when} f=0 \implies \braket{0}{0}_f	\, = \, 1 \text{which is the ground state}}
\end{align}
\end{document}
