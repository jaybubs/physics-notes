\documentclass{article}
\usepackage[margin=0.5in]{geometry}
\usepackage{amsmath}
\usepackage{physics}
\usepackage{array}
\usepackage[normalem]{ulem} 

\begin{document}
1. Pick the object
\begin{enumerate}
	\item Pick the object, copypaste the query data into a text file for reference
	\item pick at least a 10deg radius
	\item pick time as long as possible for best results
\end{enumerate}
Here's 3 clusters for this work
   \begin{tabular}{rccc}
   	\firsthline
	   & \multicolumn{3}{c}{cluster}\\
	   & Fornax & NGC4346 & Virgo \\
	   \hline
	   RA & 54.67 & 190.7 & 187.7 \\
	   DEC & -35.31 & 2.688 & 12.34\\
	   MET time & \multicolumn{3}{c}{239557417,582118346}\\
	   MeV range & \multicolumn{3}{c}{1000,1000000}\\
	   Radius & \multicolumn{3}{c}{10 degrees}
   \end{tabular}
\section{gtshit}
\subsection{pumping data together}
First it's best to link all photon files thru a common text file
\begin{verbatim}
	$ ls *PH* > binned_events.txt
\end{verbatim}
\subsection{gtselect}
next it's best to select only the relevant data - we'll be aiming to do a binned analysis so we'll name the files accordingly anyway
\begin{verbatim}
    $ gtselect evclass=128 evtype=3
    Input file: @binned_events.txt
    Output file: object_binned_filtered.fits
    RA,DEC, Radius, start met, end met: INDEF
    energy: match query
    max zenith: 90
\end{verbatim}
\subsection{gtmktime}%
this step will make sure only the only data used is taken during times the telescope was working properly
\begin{verbatim}
    $ gtmktime
    spacecraft file: spacecraft file corresponding to the chosen time period
    zenith cut: NO
    event file: object_binned_filtered.fits
    output event file: objects_binned_gti.fits
\end{verbatim}
the selection can be verified wit the gtvcut:
\begin{verbatim}
    $ gtvcut object_binned_gti.fits
    extension name: EVENTS
\end{verbatim}
\subsection{gtbin - counts map}%
this will give a neat visualisation of the raw data and shit and can be used for later comparisons and shit as well as presentations and wowing others and crap
\begin{verbatim}
    $ gtbin
    output type: CMAP
    event file: object_binned_gti.fits
    output file: object_binned_cmap.fits
    spacecraft: NONE
    X,Y axis: diameter(in deg)/resolution(in pix/deg)
    scale: see above
    coords: see query
    projection: AIT
\end{verbatim}
the resulting cmap can be visualised with either fv5.5 or (better) ds9
\subsection{gtbin - ccube}%
this will create a 3D binned cube counts map, with xy axes being ra dec as usual but z axis being energy\\
the binning of the counts map will determine the binning of the exposure calculation, this will take forever, so do it over the weekend or something, because it'll be using spacecraft data, so it's the time-log bins that will stretch it out, not the space-log bins\\
the cube has to fit in completely into the outscribed circle, a quick way to determine the side is radius*1.414, do this manipulation before decidin on the resolution
\begin{verbatim}
    $ gtbin
    output type: CCUBE
    event file: object_binned_gti.fits
    output file: object_binned_ccube.fits
    spacecraft: NONE
    X,Y axis: side(in deg)/resolution(in pix/deg)
    scale: see above
    coords: see query
    projection: AIT
    energy bins: LOG
    energy range: see query
    number of bins: idk, take something nice but not too stupid
\end{verbatim}
\subsection{XML sourcing}%
first thing first, wget/curl/copy the goddamn make4FGxml.py cos fuck doing this by hand. it'll take in a catalogue, binned events file and spit out a source file - by default it'll only spit out whatever htingy exsits, without any simulated powerlaw/dmfit/other sources, those have to be added manually later\\
the diffuse models are in \$FERMI\_DIR so echo that fucker to find out, because the python thing itself might mess up sometimes
\begin{verbatim}
    $ python make4FGxml.py gll_psc_v18.fit object_binned_gti.fits -o object_input_model.xml \
    -G $FERMI_DIR/refdata/galdiffuse/gll_iem_v07.fits -g g11_iem_v07 \
    -I $FERMI_DIR/refdata/galdiffuse/iso_P8R3_SOURCE_V2_v1.txt -i iso_P8R3_SOURCE_V2_v1 \
    -s 120 -p TRUE
\end{verbatim}
the psc file might need be downloaded cos reasons, the -s takes care of significance, -p includes point sources \\
the actual wannabe source should be added in manually (at the centre of the cluster or whatever)
it's best to fit some stupid shit like powerlaw too for a comparison, and make sure to adjust the parameters accordingly (see query re: energy range, location)\\
but the actual source for DM will be dmfit - make sure to double check the path to the function, see the fermi website for deets on how to pick channels
\subsection{gtltcube - livetimes and exposure}%
this is the fucking piece of shit that'll take forever, does what it says on the can, include a zenith cutoff to exclude rays reflected off earth:
\begin{verbatim}
    gtltcube zmax=90
    event file: object_binned_gti.fits
    spacecraft: spacecraftfile
    output file: object_binned_ltcube.fits
    step size: 0.025
    pixel size: 1
\end{verbatim}

\subsection{gtexpcube2 - exposure cube}%
here we'll calc the exposure map for the entire sky cos might as well, it'll improve shit, but also this is where the background map thing that'll be subracted is created. ofc we could just do a small portion that's just larger than our area of interest but fuck that, the time waiting difference between this and the entire sky ain't that much.\\
just make sure that the geometry is the same as for the ccube - ie make sure the degs/pixel is the same, and use 360/that thing and 180/that thing to get the values in pixel for the axes
\begin{verbatim}
    $ gtexpcube2
    livetime file: object_binned_ltcube.fits
    counts map: none
    output: object_binned_expcube.fits
    response: P8R3_SOURCE_V2
    size x,y, scale: see above
    coordinates: see query
    projection: AIT
    energies: match ccube
\end{verbatim}

\subsection{gtsrcmaps - creating sourcemaps}%
here we'll make use of the fkn xml finally creating a file that will be passed on to the binned likelihood, it'll take all the point sources from the catalog as well as the diffusion ones and apply it onto the exposure map
\begin{verbatim}
    $ gtsrcmaps
    hypercube file: object_binned_ltcube.fits
    counts file: object_binned_ccube.fits
    exposure file: object_expcube.fits
    source file: object_input_model.xml
    response: CALDB
\end{verbatim}
caveat: if this shit fails (especially file not found shit), instead of googling like mad, try to remake the input file a couple of times
\subsection{gtlike - likelihood}%
fucking finally, alright, first of all, it's a fucking piece of shit running this on wsl, not because of performance, quite the contrary that's fine, but due to the fucking goddamn impossibility to display the fucking plot right away, so have fun with the output\\
\sout{oh brilliant idea - run xterm, that should work} nope, but clearly running this directly in lunex or windoze as opposed to wsl should be ok, maybe i'll come back to this later or at least figure out how to plot manually ffs\\
now, this is straight forward cos we have all the inputs:
\begin{verbatim}
    $ gtlike refit=yes plot=yes sfile=object_binned_output.xml \
    results=object_results_type.dat specfile=object_spectra_type.fits [> object_log.txt]
    stats: BINNED
    counts file: object_srcmaps.fits
    exposure file: object_expcube.fits
    hypercube file: object_ltcube.fits
    source file: object_input_model.xml
    response: CALDB
    optimizer: NEWMINUIT
\end{verbatim}
not much to add, refit will offer a chance to edit the source and will attempt refitting, plot will plot (when possible), sfile is the output name, other shit gets logged into default names

\subsection{gtmodel - creating a model map}%
now that the results and/or model source have been chucked into a binned output, we'll use that to create a new plot based off of that
\begin{verbatim}
    $ gtmodel
    source maps: object_srcmaps.fits
    source model: object_binned_output.xml
    output: object_model_map.fits
    response: CALDB
    exposure cube: object_ltcube.fits
    binned exp map: object_expcube.fits
\end{verbatim}
This will create a sweet af looking map to look at, but ofc it makes sense to compare this to the original. So we'll redo the entire restricted view thing again but with the original counts.
\begin{verbatim}
    $ gtbin
    output type: CMAP
    event file: object_binned_gti.fits
    output file: object_binned_cmap_small.fits
    spacecraft: NONE
    X, Y,axis: see ccube
    scale: see ccube
    coords: see query
    projection: STG
\end{verbatim}
theree's one final step in computing the residual map as according to the fermi tutorial. however this shit doesn't work in conda for some reason, anyway, here's the log in case it gets fixed and/or you'll be using it directly in lunix. it'll get the residual map to check for incostincencies.
\begin{verbatim}
    $ farith
    1st FITS: object_binned_cmap_small.fits
    2nd FITS: object_madel_map.fits
    Output name: object_residual_model.fits
    Operation: SUB
\end{verbatim}

\end{document}
